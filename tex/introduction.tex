\section{はじめに}
\label{sec:introduction}
知識ベースとは,様々な知識が蓄積されたデータベースである.そこには,世界に関する汎用的な知識から専門的な知識までが蓄積されている.代表的な知識ベースには,Wikipediaの情報を基にしたDBpedia,Wikidata,YAGOなどがある.人間や機械は知識ベースを使うことで,質問に答えたり,新たな知識を発見することができる.
%例えば,検索エンジンでキーワード検索を実行すると,検索結果の画面右側に,キーワードが指し示すエンティティに関する情報が提示される.知識ベースを用いて,エンティティ検索を実行した結果である(図~\ref{fig:entitysearch}).
%\begin{figure}[h]
%\centering
%\includegraphics[width=3in]{image/entitysearch.png}
%\caption{
%検索エンジンにおける知識ベースの利用例
%}
%\label{fig:entitysearch}
%\end{figure}

知識ベースの記述には,Resource Description Framework(RDF)が用いられる.RDFとは,リソースに関する情報を記述する方法である.RDFでは,Universal Resource Identifier(URI)で識別されるものすべてをリソースとして扱う.世の中のあらゆるエンティティはURIを付けることで,リソースとして記述することができる.あるリソースについての1つの情報は,主語(Subject),述語(Predicate),目的語(Object)から構成される3つ組(トリプル)のグラフ構造で記述される(図~\ref{fig:RDFimage}).主語は情報を記述される対象のリソースを示し,述語は主語に関する情報のプロパティを定義する.そして,目的語には述語の対象となる値を格納する.主語と述語はURIで記述し,目的語はURIもしくは数値や文字列などのリテラルで記述する.
%
\begin{figure}[h]
\centering
\includegraphics[width=3in]{image/RDFimage.png}
\caption{\small
RDFの例
}
\label{fig:RDFimage}
\end{figure}
%

RDFで記述された知識ベースに対して検索を行うにはいくつかの方法がある.主要なものの1つは問合せ言語SPARQLを用いた検索である.SPARQLの文法に従って,トリプルの条件を指定することで,情報を取り出すことができる.しかし,一般ユーザにとって,SPARQL検索を行うハードルは高い.なぜなら,まず,SPARQLの文法を理解し習得する必要がある.そして,検索対象の知識ベースで定義されているプロパティやエンティティについて理解する必要があるためである.知識ベースは様々な種類のプロパティやエンティティが存在する複雑なグラフ構造になっているため,特に後者の理解は難しい.もう1つの方法として,キーワード検索がある.キーワード検索は,キーワードを入力することで検索が行えるため,前提知識が不要な検索手法である.結果には,キーワードに関するエンティティのランキングが返却される.しかし,様々な種類のエンティティが混在しているため,ユーザは欲しい情報を判断しづらい.また,ユーザはどのような種類のエンティティがあるのか把握していない場合や,そもそもどのような種類のエンティティが欲しいのかわかっていない場合もある.したがって,前提知識が不要という点で,キーワード検索は有効な手段だが,情報を整理して提示する必要があると考える.

こうした課題を解決する検索方法としてファセット検索がある.ファセット検索では,検索対象のエンティティを様々な切り口(ファセット)で絞り込む.ユーザは結果を確認し,ファセットを切り替えたり,組み合わせたりすることで,意図する結果を得るまで,対話的に検索を行うことができる.したがって,知識ベースのプロパティやエンティティの種類などの知識を持たない場合でも,検索を容易に実行できる.これまでに提案されてきたRDFの知識ベースに対するファセット検索システムの多くは,エンティティが持つプロパティをファセットとして利用してきた.しかし,プロパティそれ自体も数多くの種類が存在しているため,必要となるプロパティ(ファセット)を見つけ出すことは簡単ではない.この課題を解決するには,数多くあるプロパティから興味のあるプロパティを見つけ易くすることが必要である.そのために,プロパティをその主語と目的語との関係性によってクラスタリング した結果をプロパティのファセットとして利用することを提案する.そして,興味のあるプロパティで関係付けられたエンティティ(トリプル)を検索できる,プロパティ指向の新しいファセット検索システムの開発を提案する.

本稿では,プロパティ指向のファセット検索システムの全体概要と,その機能要素であるプロパティファセットのためのプロパティのクラスタリング について報告する.