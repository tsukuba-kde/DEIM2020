\section{予備実験}
\label{sec:experiments}
プロパティ指向のファセット検索システムの機能要素となるプロパティに関するファセットの生成とその結果について目視による初期的な確認を行うことを目的として,次の2種類のプロパティのクラスタリング 結果についての確認を行なった.
\begin{enumerate}
	\item 上位プロパティによるクラスタリング
	\item 階層型クラスタリング によるクラスタリング
\end{enumerate}
また,プロトタイプシステムを用いて,階層型クラスタリング によって生成したプロパティに関するファセットを用いた検索結果の確認を行った.
%
\subsection{クラスタリング }
\subsubsection{上位プロパティによるクラスタリング }
知識ベースの語彙の体系に基づいて,プロパティをクラスタリング する.プロパティは,RDFスキーマと呼ばれる語彙を定義する基本的な仕組みによって,そのプロパティの性質が定義されている.その中で,上位プロパティを定義するために使用されるrdfs:subPropertyOfというプロパティを利用して,各プロパティを上位プロパティでクラスタリング  した.クラスタの名称は上位プロパティの名称である.
\subsubsection{階層型クラスタリング によるクラスタリング }
実際のデータの関係性に基づいて,プロパティをクラスタリング する.クラスタリング の手法は群平均法による階層型クラスタリング を用いた.クラスタリング に用いたプロパティの類似度には,プロパティの主語に対するJaccard係数と目的語に対するJaccard係数の平均値を計算し,Jaccard距離に変換した値を用いた.したがって,各クラスタの要素は,主語と目的語の重複度合いが類似するプロパティとなっている.クラスタの分割はデンドログラムを確認し,分割の閾値を0.990として実行した.クラスタの名称は,クラスタの要素数が1つしかない場合は,その要素の名称をそのまま用いた.また,要素数が複数ある場合は,各要素の内容を確認して便宜的に付けた.
%
\subsection{実験環境}
\label{sec:environment}
提案手法をPython 3.7.3で実装し,Intel(R) Core\texttrademark i7-7700 3.60 GHz CPU, 32 GB RAMを搭載したUbuntu 18.04.3 LTSで実験を行った.
%
\subsection{データセット}
\label{sec:datasets}
本実験では,DBpedia 2016-10\footnote{https://wiki.dbpedia.org/downloads-2016-10}\ のInstance Typesにおいて,クラスがUniversity、Company、Scientist、Politician、Astronautであるエンティティを抽出し,それらのエンティティを記述するトリプルをMappingbased Objectsのデータセットから抽出し,データセットとした.データセットの統計情報を図~\ref{fig:datasetStatistics}に示す.
%
\begin{figure}[h]
\centering
\includegraphics[width=3in]{image/datasetStatistics.png}
\caption{\small
実験に用いたデータセットの統計情報
}
\label{fig:datasetStatistics}
\end{figure}
%
\subsection{上位プロパティによるクラスタリング 結果}
\label{sec:upperPropertyClusteringResult}
上位プロパティを使うと,101個のプロパティを12個のクラスタにまとめられた(図~\ref{fig:upperPropertyClusters}).結果から,上位プロパティには名前空間の異なるプロパティも定義されていることがわかる.

また,図~\ref{fig:partOfUpperPropertyCluster}は,最もサイズの大きなクラスタである``sameSettingAs'' に含まれるプロパティの一覧である.``sameSettingAs'' のプロパティは,rdfs:comment(リソースの説明文を記述するプロパティ)によると,``A relation between two entities participating in a same Situation; e.g., 'Our company provides an antivenom service' (the situation is the service, the two entities are the company and the antivenom).'' とあるが,直感的には,``sameSettingAs'' と図~\ref{fig:partOfUpperPropertyCluster}のプロパティの関係性を理解することは難しいと思われる.
%
\begin{figure}[h]
\centering
\includegraphics[width=3in]{image/upperPropertyClusters.png}
\caption{\small
上位プロパティによるクラスタリング 結果
}
\label{fig:upperPropertyClusters}
\end{figure}
%
\begin{figure}[h]
\centering
\includegraphics[scale=0.5]{image/partOfUpperPropertyCluster.png}
\caption{\small
``sameSettingAs''に属すプロパティの一覧
}
\label{fig:partOfUpperPropertyCluster}
\end{figure}
%
\subsection{階層型クラスタリング によるクラスタリング 結果}
\label{sec:hierachicalClustersResult}
階層型クラスタリング では,101個のプロパティについて23個のクラスタを生成した(図~\ref{fig:hierachicalClusters}).クラスタに含まれる要素を確認すると,一定の領域に関係するエンティティに対して関係付けられるプロパティでまとまっていることが確認できた.図~\ref{fig:partOfHierachicalCluster}は,UniversityクラスタとAcademicsクラスタのプロパティの一覧である.大学という物理的・組織的なエンティティに関係付けられるプロパティと,学者や研究といったエンティティに関係付けられるプロパティに分けることができている.
%
\begin{figure}[h]
\centering
\includegraphics[width=3in]{image/hierachicalClusters.png}
\caption{\small
階層型クラスタリング によるクラスタリング 結果
}
\label{fig:hierachicalClusters}
\end{figure}
%
\begin{figure}[h]
\centering
\includegraphics[scale=0.5]{image/partOfHierachicalCluster.png}
\caption{\small
``University''クラスタと``Academics''クラスタのプロパティ一覧
}
\label{fig:partOfHierachicalCluster}
\end{figure}
%
\subsection{プロパティに関するファセットを用いた検索}
主語のエンティティに対して,``Ivy League''でキーワード検索を行うと仮定し,検索結果を確認する.

\subsubsection{キーワード検索結果}
キーワード検索の結果,主語のエンティティのドキュメントに``Ivy League''を含むトリプルを359個取得した.内訳は,主語のエンティティの種類数が39個,プロパティの種類数が26個,目的語のエンティティの種類数が137個である.

キーワード検索結果に対応するプロパティに関するファセット(Predicate Type)は,図~\ref{fig:propertyFacet}の赤枠のように生成されている.また,主語のファセット(Subject Type)には2つのクラスが,目的語のファセット(Object Type)には,21個のクラスがファセットキーとして生成されていることを確認した.この結果を踏まえて,ユーザはPredicate TypeからUniversityを選択し,大学に関係するようなエンティティ間の関係性について探索すると仮定する.
%
\begin{figure}[h]
\centering
\includegraphics[width=3in]{image/propertyFacet.png}
\caption{\small
キーワード検索結果の一部と対応するプロパティファセット
}
\label{fig:propertyFacet}
\end{figure}
%
\subsubsection{ファセット検索結果}
Universityを選択した結果,215個のトリプルを取得した.内訳は,主語のエンティティの種類数が30個,プロパティの種類数が10個,目的語のエンティティの種類数が89個である.プロパティは大学に関係するようなプロパティのみになっていることが確認できる(図~\ref{fig:facetResult}).また,この時,主語のエンティティのクラスは1種類(University)となり,目的語のエンティティのクラスは15種類となっていることを確認した.このようにして,ユーザは,大学に関係するようなプロパティに絞り込んだ上で,Universityのクラスに属す主語のエンティティと他のクラスに属す目的語のエンティティとの関係性を探索することができる.
%
\begin{figure}[h]
\centering
\includegraphics[width=3.5in]{image/facetResult.png}
\caption{\small
Predicate Type ``University''のファセット検索結果の一部
}
\label{fig:facetResult}
\end{figure}
%


