\section{提案手法}
\label{sec:proposal}
知識ベースに対するプロパティ指向のファセット検索システムとして,図~\ref{fig:architecture} のシステム概要を提案する.プロパティ指向のファセット検索システムの目的は,興味のあるエンティティをキーワード検索し,関心のあるプロパティ集合をプロパティファセットから選択することで,それらのプロパティで関係付けられたエンティティ(トリプル)を見つけることである.ユーザの操作とシステムで実行される処理は次の通りである.
\begin{enumerate}
	\item ユーザはキーワード検索を実行する.システムは,キーワードをエンティティのドキュメントに含むエンティティの部分集合を生成し,それらが主語または目的語に位置付けられているトリプル集合を主語あるいは目的語のランク値降順で返却する.また,返却したトリプル集合に含まれるプロパティについて,プロパティファセットを提示する.
	\item ユーザはプロパティファセットから興味のあるファセットキーを選択する.システムは,選択されたファセットキーに含まれるプロパティを持つトリプル集合を返却する.
\end{enumerate}

1におけるキーワード検索の対象は,エンティティを説明するリテラルとエンティティを1つのドキュメントとみなしたドキュメント化されたエンティティである.本システムのポイントは,プロパティファセットのような柔軟なファセット生成を可能とするために,知識ベースから独立してデータベースを設計することにある.以降で,RDFデータベース,エンティティデータベース,ファセットデータベースを説明する.
%
\begin{figure}[h]
\centering
\includegraphics[width=3in]{image/architecture.png}
\caption{\small
システム概要
}
\label{fig:architecture}
\end{figure}
%
\subsection{RDFデータベース}
RDFデータベースは,キーワード検索とファセット検索に必要となる知識ベースのデータを格納する.本稿では,エンティティのランク値を計算するためのRDFデータ,ファセットを生成するためのRDFデータ,検索対象となるドキュメント化されたエンティティを生成するためのRDFデータ,の3種類に大別できる.
%
\subsection{エンティティデータベース}
エンティティデータベースは,エンティティを主キーとして,検索対象のドキュメント化されたエンティティ,ランク値をタプルとしたテーブルを持つ.RDFデータベースのエンティティのランク値を計算するためのデータをPageRankで処理し,結果をエンティティとランク値をタプルとしたテーブルに格納する.また,ドキュメント化されたエンティティを生成するためのRDFデータに対して,主語のエンティティと目的語のリテラルを1つのドキュメントとする処理を行う.結果として,エンティティとドキュメント化されたエンティティをタプルとしたテーブルに格納する.2つのテーブルを結合演算し,エンティティテーブルを作成する.
%
\subsection{トリプルデータベース}
クラスタリング 対象のプロパティが述語であるトリプルをタプルとしたテーブルを持つ.なお,目的語にリテラルを含むトリプルは除いている.
%
\subsection{ファセットデータベース}
ファセットデータベースは,テーブル``プロパティファセット''とテーブル``ファセットキー''を持つ.プロパティファセットは,ファセットの種類を示す番号,ファセットの内容を示すキー,プロパティのURIをタプルとして持つ.ファセットキーは,ファセットの種類を示す番号,ファセットキー,キーの内容を示すラベルをタプルとして持つ.ファセットの種類を示す番号は,ファセットの種類の拡張に対応するためである.また,ファセットキーは,プロパティのクラスタに該当する.本稿では,プロパティファセットを群平均法による階層型クラスタリング によって生成した.プロパティ間の距離は,Jaccard係数を変換したJaccard距離である.Jaccard係数は,各プロパティの主語に関するJaccart係数と目的語に関するJaccard係数の平均値である.この手法を適用した理由として,プロパティはエンティティとエンティティ(またはリテラル)の関係性として機能するため,プロパティ間で同じ主語や目的語を共有している可能性があるからである.別の方法として,RDFスキーマで定義されるdomain(主語に期待されるエンティティのクラス)やrange(目的語に期待されるエンティティのクラスやリテラルのデータ型)も考えられるが,あくまでも期待値であり実際には異なるクラスのエンティティに関係付けられていることもあるため,実際のデータに基づいてクラスタリング を行うことが有効と考えた.

このように,知識ベースから独立してデータベースを設計することで,実際のRDFデータの状態に合わせて必要なファセット生成の処理を柔軟に実行することができる.