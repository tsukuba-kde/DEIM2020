\section{関連研究}
\label{sec:related}
\subsection{RDFの検索結果に対するランキング}
RDFの検索結果の有用性を向上させることを目的に,Ichinoseら~\cite{Ichinose2013RankingTR}はDBpediaに対してPageRankを用いてエンティティを評価する手法を提案している.主語と目的語をノードとし,述語をエッジとしてPageRankを計算し,エンティティの重要度を評価している.この研究の実験では,検索対象のエンティティが既知,あるいはSPARQLクエリにおいてトリプルの条件が設定されており,対象ユーザはSPARQLや対象データのDBpediaに関する知識を持っていることを前提としている.本研究では,エンティティの評価にはPageRankを用いるが,ユーザはSPARQLや知識ベースに関する知識を有していないケースを想定しているため,前提条件が異なる.

\subsection{RDFに対するキーワード検索}
RDFに対するキーワード検索を行う研究として,奥村ら~\cite{okumura}のObjectRankと適合フィードバックを用いた研究がある.検索対象となるエンティティはリテラルを目的語に持つトリプルを持つことを前提にして,エンティティをリテラルを含めたドキュメントとみなしている.本研究でもこの考え方に則る.しかし,検索対象となるエンティティの種類がシステム構築者などによってあらかじめ設定されることを前提とした手法であるため,知識ベースに含まれる多様なエンティティの全てに応える方法ではない.また,ユーザの検索意図に応える方法として,適合フィードバックを採用している点も,ファセット検索とはアプローチが異なる.

\subsection{知識ベースに対するファセット検索}
知識ベースに対するファセット検索の研究は数多く行われている.本稿の対象データセットであるDBpediaに関しては,Brunkら~\cite{Brunk2011tFacetHF}が,DBpediaのオントロジーを利用して,階層的なタイプ情報をファセットとして選ばせるtFacetを提案している.Arenasら~\cite{Arenas:2014:SSF:2567948.2577011}は,知識ベースのYAGOに対してファセット検索を行うSemFacetを提案している.SemFacetでは,ファセットにRDFにおける目的語やプロパティを使用し,検索対象にエンティティ(URI)を設定している.エンティティに関する情報が疎である場合があることが課題とされるが,OWL2のオントロジーを用いて推論することによって解決することを提案している.Papadakosら~\cite{Papadakos2014HippalusPF}は,ファセットをランキングするHippalusというシステムを提案している.Hippalusでは,ユーザが検索プロセスの中でファセットを評価し,そのファセットの評価に基づいてファセットをランキングして返す.これにより,ユーザの好みに合わせたファセット検索が行えるとしている.Bastら~\cite{Bast:2014:EAF:2567948.2577016}は,知識ベースの1つであったFreebaseのファセット検索を提案している.主な特徴は,ファセット検索の利便性を向上させるために,データセットそのものを編集したことにある.具体的には,データセットに含まれる冗長なエンティティやプロパティを削除や統一,タイプのタキソノミー(分類)の編集などを行なっている.Wikidataに関しては,Moreno-Vegaら~\cite{10.1007/978-3-030-00671-6_18}が大規模な知識ベースに対するクエリの高速化を目指したGraFaを提案している.

上記の研究では,既存のプロパティをエンティティのファセットとして利用し,エンティティを検索することを目的としている.本研究では,まず関心のあるプロパティを見つけ出すことを重視する.そのために,プロパティに関するファセットを生成する.そして,そのプロパティファセットに基づいて,関係付けられているエンティティの検索を可能にするシステムを目指している点が異なる.