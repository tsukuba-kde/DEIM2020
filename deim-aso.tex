\documentclass{deimj}
\usepackage[dvipdfm]{graphicx}
%\usepackage{latexsym}
%\usepackage{txfonts}
%\usepackage[fleqn]{amsmath}
%\usepackage[psamsfonts]{amssymb}
%\usepackage[deluxe]{otf}

% 印刷位置調整 %
% 必要に応じて値を変更してください.
\hoffset -10mm % <-- 左に 10mm 移動
\voffset -10mm % <-- 上に 10mm 移動

\newcommand{\AmSLaTeX}{%
 $\mathcal A$\lower.4ex\hbox{$\!\mathcal M\!$}$\mathcal S$-\LaTeX}
\newcommand{\PS}{{\scshape Post\-Script}}
\def\BibTeX{{\rmfamily B\kern-.05em{\scshape i\kern-.025em b}\kern-.08em
 T\kern-.1667em\lower.7ex\hbox{E}\kern-.125em X}}

\papernumber{DEIM2020 D5-3(day2 p33)}

\jtitle{知識ベースに対するプロパティ指向のファセット検索システムに関する研究}
%\jsubtitle{サブタイトル} <- サブタイトルを付けたいときはこの行の先頭の % を取る
\authorlist{%
 \authorentry[aso@kde.cs.tsukuba.ac.jp]{阿曽 太郎}{Taro Aso}{筑波大学}%
 \authorentry[amagasa@cs.tsukuba.ac.jp]{天笠 俊之}{Toshiyuki Amagasa}{CCS}%
 \authorentry[kitagawa@cs.tsukuba.ac.jp]{北川 博之}{Hiroyuki Kitagawa}{CCS}%
}
\affiliate[筑波大学]{筑波大学システム情報工学研究科\hskip1zw
  〒305--8573 茨城県つくば市天王台1-1-1}
 {Graduate School of Systems and Information Engineering, Tsukuba University\\
  1--1--1 Tennodai, Tsukiba-shi, Ibaraki 305--8573, Japan}
\affiliate[CCS]{筑波大学計算科学研究センター\hskip1zw
  〒305--8573 茨城県つくば市天王台1-1-1}
 {Center for Computational Science,\\
  1--1--1 Tennodai, Tsukiba-shi, Ibaraki 305--8573, Japan}

%\MailAddress{$\dagger$hanako@deim.ac.jp,
% $\dagger\dagger$\{taro,jiro\}@jforum.co.jp}

\begin{document}
\pagestyle{empty}
\begin{jabstract}
知識ベースとは様々な知識が蓄積されたデータベースである.本研究では,その中でもRDFで記述された知識ベースに焦点を当てる.一般に,RDFによる知識ベースのデータ構造は複雑であるため,専門知識を持たないユーザが簡単に検索を行うには,ファセット検索が有効であることが知られている.ファセットとはデータの切り口であり,ユーザはファセットの選択・解除を繰り返すことで対話的な検索を実行できる.本論文では,RDFの述語(プロパティ)について,主語と述語との構造的な関係に着目してクラスタリング することで,プロパティに関するファセットを生成することを提案する.
\end{jabstract}
\begin{jkeyword}
知識ベース,RDF,ファセット検索,クラスタリング
\end{jkeyword}
\maketitle

\vspace{-5cm}
\section{はじめに}
\label{sec:introduction}
知識ベースとは,様々な知識が蓄積されたデータベースである.代表的な知識ベースには,Wikipediaの情報を基にしたDBpediaやWikidata,YAGOなどがある.人間や機械は知識ベースを使うことで,質問に答えたり,新たな知識を発見することができる.

知識ベースの記述には,Resource Description Framework(RDF)が用いられる.RDFとは,リソースに関する情報を記述する方法である.RDFでは,Universal Resource Identifier(URI)で識別されるものすべてをリソースとして扱う.世の中のあらゆるエンティティはURIを付けることで,リソースとして記述することができる.あるリソースについての1つの情報は,主語(Subject),述語(Predicate),目的語(Object)から構成される3つ組(トリプル)のグラフ構造で記述される(図~\ref{fig:RDFimage}).主語は情報を記述される対象のリソースを示し,述語は主語に関する情報のプロパティを定義する.そして,目的語は述語の対象である.主語と述語はURIで記述し,目的語はURIもしくは数値や文字列などのリテラルで記述する.
%
\begin{figure}[h]
\centering
\includegraphics[width=3in]{image/RDFimage.png}
\caption{\small
RDFの例
}
\label{fig:RDFimage}
\end{figure}
%

RDFで記述された知識ベースに対して検索を行うにはいくつかの方法がある.主要なものの1つは問合せ言語SPARQLを用いた検索である.SPARQLの文法に従って,トリプルの条件を指定することで,情報を取り出すことができる.しかし,一般ユーザにとって,SPARQL検索を行うハードルは高い.なぜなら,まず,SPARQLの文法を理解し習得する必要がある.そして,検索対象の知識ベースで定義されているプロパティやエンティティについて理解する必要があるためである.知識ベースは様々な種類のプロパティやエンティティが存在する複雑なグラフ構造になっているため,特に後者の理解は難しい.もう1つの方法として,キーワード検索がある.キーワード検索は,キーワードを入力することで検索が行えるため,前提知識が不要な検索手法である.結果には,キーワードに関するエンティティのランキングが返却される.しかし,様々な種類のエンティティが混在しているため,ユーザは欲しい情報を判断しづらい.また,ユーザはどのような種類のエンティティがあるのか把握していない場合や,そもそもどのような種類のエンティティが欲しいのかわかっていない場合もある.したがって,前提知識が不要という点で,キーワード検索は有効な手段だが,情報を整理して提示する必要があると考える.

こうした課題を解決する検索方法としてファセット検索がある.ファセット検索では,検索対象のエンティティを様々な切り口(ファセット)で絞り込む.ユーザは結果を確認し,ファセットを切り替えたり,組み合わせたりすることで,意図する結果を得るまで,対話的に検索を行うことができる.したがって,知識ベースのプロパティやエンティティの種類などの知識を持たない場合でも,検索を容易に実行できる.これまでに提案されてきたRDFの知識ベースに対するファセット検索システムの多くは,エンティティが持つプロパティをファセットとして利用してきた.しかし,プロパティそれ自体も数多くの種類が存在しているため,必要となるプロパティ(ファセット)を見つけ出すことは簡単ではない.この課題を解決するには,数多くあるプロパティから興味のあるプロパティを見つけ易くすることが必要である.そのために,プロパティをその主語と目的語との関係性によってクラスタリング した結果をプロパティに関するファセットとして利用することを提案する.そして,プロパティに関するファセットを利用して,関係するエンティティ集合(トリプル集合)を検索できる,プロパティ指向の新しいファセット検索システムを提案する.

本稿では,プロパティ指向のファセット検索システムの全体概要,プロトタイプシステムのインターフェース,機能要素であるファセットに関してプロパティのクラスタリング について報告する.

\vspace{-3mm}
\section{前提知識}
本節では,前提知識として,ファセット検索について説明する.
%本節では,RDF(Resource Description Framework),ファセット検索,Latent Dirichlet Allocation(LDA)について述べる.
%
%\subsection{RDF}
%RDFとは,LODのデータを記述するためのフレームワークである.RDFでは,Universal Resource Identifier(URI)で識別されるものすべてをリソースとして扱う.あるリソースについての1つの情報は,主語(Subject),述語(Predicate),目的語(Object)から構成される3つ組(トリプル)のグラフ構造で記述される.主語は情報を記述される対象のリソースを示し,述語は主語に関する情報のプロパティを定義する.そして,目的語には述語の対象となる値を格納する.主語と述語はURIで記述し,目的語はURIもしくはリテラルで記述する.LODのデータセットとは,トリプルの集合から成る.RDFの例を図~\ref{fig:RDFimage}に示す.なお,図において,楕円がURI,矩形がリテラルである.
%
%\begin{figure}[h]
%\centering
%\includegraphics[width=3.5in]{image/RDFimage.png}
%\caption{RDFの例}
%\label{fig:RDFimage}
%\end{figure}
\subsection{ファセット検索}
ファセット検索とは,探索的検索における手法の1つである.その特徴は,検索対象のエンティティ集合を,様々な切り口(ファセット)によって絞り込むことで,意図するエンティティを発見しようとするところにある.例えば,E-CommerceのAmazonの商品検索サイトもファセット検索である.基本的なインターフェースとして図~\ref{fig:InterfaceImage2}を例にあげる.ユーザはキーワード検索によって,Extensionに初期の検索結果を,Transition Markerに初期のファセットを得る.その状態から,ファセットの選択や,選択したファセットの解除を対話的に繰り返して,検索結果を洗練させていく.
%
\begin{figure}[h]
\centering
\includegraphics[width=3in]{image/InterfaceImage2.png}
\caption{\small
ファセット検索のインターフェースイメージ
}
\label{fig:InterfaceImage2}
\end{figure}
%
%\subsection{LDA}
%Latent Dirichlet Allocation(LDA)とは,1つの文書が複数のトピックを持つと仮定するトピックモデルの1つである.トピックモデルとは,Bag-of-Words(BOW)で表現された文書集合を生成するための確率モデルである.BOWとは,文書を単語の多重集合で表現する方法である.LDAは,文書が持つトピックがわかるという特徴から,トピックが似ている文書を見つけることや,トピックに基づいた文書分類に活用されている.

%LDAによる文書集合の生成プロセスについて述べる.BOWで表現される文書dはN単語のベクトル$\mathbf{w}_d = (w_{d1}, ..., w_{dN})$ で表現する.BOWで表現されたD個の文書集合を $\mathbf{W} = \{ \mathbf{w}_1, ..., \mathbf{w}_D \}$とする.それらの語彙の集合を $V=\{ v_1, v_2, ..., v_{|V|} \}$とする.K個のトピックの各トピックkは単語分布$\phi_k = (\phi_{k1}, ..., \phi_{k|V|})$ を持つ.このとき,$\phi_{kv} = p(v|\phi_k)$ はトピックkで語彙vが生成される確率 ($\phi_{kv} \ge 0, \textstyle\sum_{v=1}^{|V|} \phi_{kv} = 1$) を表す.文書ごとにトピック分布 $\theta_d = (\theta_{d1}, ..., \theta_{dK})$ を仮定するので,それに従い,文書dの各単語にトピック$z_{dn}$ が割り当てられる.なお,$\theta_{dk} = p(k|\theta_d)$ は文書dの単語にトピックkが割り当てられる確率である ($\theta_{dk} \ge 0, \textstyle\sum_{k=1}^K \theta_{dk} = 1$).そして,割り当てられたトピックの単語分布 $\phi_{z_{dn}}$ に従って単語が生成される.この文書集合の生成過程をまとめると下記の通りである.
%\begin{enumerate}
%\item For トピック k = 1, ..., K
%	\begin{enumerate}
%	\item 単語分布を生成 $\phi_k$ 〜 Dirichlet($\beta$)
%	\end{enumerate}
%\item For 文書 d = 1, ..., D
%	\begin{enumerate}
%	\item トピック分布を生成 $\theta_d$ 〜 Dirichlet($\alpha$)
%	\item For 単語 n = 1, ..., N
%		\begin{enumerate}
%		\item トピックを生成 $z_{dn}$ 〜 Multi($\theta_d$)
%		\item 単語を生成 $w_{dn}$ 〜 Multi($\phi_{z_{dn}}$)
%		\end{enumerate}
%	\end{enumerate}
%\end{enumerate}

%ここで,トピックごとの単語分布 $\phi_k$ および文書ごとのトピック分布 $\theta_d$ は多項分布のパラメータであり,その共役事前分布であるディリクレ分布から生成されると仮定している.また,ディリクレ分布の$\alpha$,$\beta$はハイパーパラメータである.

%式にすると以下の通りである.\[ p(w_i|d) = \sum_{k=1}^K p(w_i|z_k)p(z_k|d) \]
%1つのトピック $z_j$, $1\le j \le K$ は $|V|$ についての多項分布で表現され,$p(w_i|z_j),\textstyle\sum_{i=1}^{|V|} p(w_i|z_j) = 1$ である.LDAは単語集合を2段階で生成する:文書集合からトピックを生成し,トピックから単語集合を生成する.より正確には,文書の単語分布

\vspace{-3mm}
\section{関連研究}
\label{sec:related}
\subsection{RDFの検索結果に対するランキング}
RDFの検索結果の有用性を向上させることを目的に,Ichinoseら~\cite{Ichinose2013RankingTR}はDBpediaに対してPageRankを用いてエンティティを評価する手法を提案している.主語と目的語をノードとし,述語をエッジとしてPageRankを計算し,エンティティの重要度を評価している.この研究の実験では,検索対象のエンティティが既知,あるいはSPARQLクエリにおいてトリプルの条件が設定されており,対象ユーザはSPARQLや対象データのDBpediaに関する知識を持っていることを前提としている.本研究では,エンティティの評価にはPageRankを用いるが,ユーザはSPARQLや知識ベースに関する知識を有していないケースを想定しているため,前提条件が異なる.

\subsection{RDFに対するキーワード検索}
RDFに対するキーワード検索を行う研究として,奥村ら~\cite{okumura}のObjectRankと適合フィードバックを用いた研究がある.検索対象となるエンティティはリテラルを目的語に持つトリプルを持つことを前提にして,エンティティをリテラルを含めたドキュメントとみなしている.本研究でもこの考え方に則る.しかし,検索対象となるエンティティの種類がシステム構築者などによってあらかじめ設定されることを前提とした手法であるため,知識ベースに含まれる多様なエンティティの全てに応える方法ではない.また,ユーザの検索意図に応える方法として,適合フィードバックを採用している点も,ファセット検索とはアプローチが異なる.

\subsection{知識ベースに対するファセット検索}
知識ベースに対するファセット検索の研究は数多く行われている.本稿の対象データセットであるDBpediaに関しては,Brunkら~\cite{Brunk2011tFacetHF}が,DBpediaのオントロジーを利用して,階層的なタイプ情報をファセットとして選ばせるtFacetを提案している.Arenasら~\cite{Arenas:2014:SSF:2567948.2577011}は,知識ベースのYAGOに対してファセット検索を行うSemFacetを提案している.SemFacetでは,ファセットにRDFにおける目的語やプロパティを使用し,検索対象にエンティティ(URI)を設定している.エンティティに関する情報が疎である場合があることが課題とされるが,OWL2のオントロジーを用いて推論することによって解決することを提案している.Papadakosら~\cite{Papadakos2014HippalusPF}は,ファセットをランキングするHippalusというシステムを提案している.Hippalusでは,ユーザが検索プロセスの中でファセットを評価し,そのファセットの評価に基づいてファセットをランキングして返す.これにより,ユーザの好みに合わせたファセット検索が行えるとしている.Bastら~\cite{Bast:2014:EAF:2567948.2577016}は,知識ベースの1つであったFreebaseのファセット検索を提案している.主な特徴は,ファセット検索の利便性を向上させるために,データセットそのものを編集したことにある.具体的には,データセットに含まれる冗長なエンティティやプロパティを削除や統一,タイプのタキソノミー(分類)の編集などを行なっている.Wikidataに関しては,Moreno-Vegaら~\cite{10.1007/978-3-030-00671-6_18}が大規模な知識ベースに対するクエリの高速化を目指したGraFaを提案している.

上記の研究では,既存のプロパティをエンティティのファセットとして利用し,エンティティを探索できるようにすることを目的としている.一方で,本研究では,エンティティ間の関係性および関係付けられているエンティティの探索を重視する.そのために,エンティティ間の関係性を定義するプロパティに関するファセットを生成し,プロパティの探索を従来より容易にすることを目指している点が新しい.

\vspace{-3mm}
\section{提案手法}
\label{sec:proposal}
知識ベースに対するプロパティ指向のファセット検索システムとして,図~\ref{fig:architecture} のシステム概要を提案する.プロパティ指向のファセット検索システムの目的は,興味に関してキーワード検索して得たトリプルの集合に対して,エンティティ間の関係性を示すプロパティに関するファセットを整備することで,関係性に基づいてエンティティの集合(トリプルの集合)を検索できるようにすることである.ユーザの操作とシステムで実行される処理は次の通りである.
\begin{enumerate}
	\item ユーザはキーワード検索を実行する.システムは,キーワードをエンティティのドキュメントに含むエンティティについて,それらが主語または目的語に位置付けられているトリプル集合を主語あるいは目的語のランク値降順で返却する.また,返却したトリプル集合に含まれる主語,述語,目的語についてファセットを提示する.
	\item ユーザはファセットから探索したいファセットキーを選択する.システムは,選択されたファセットキーに対応するトリプル集合を返却する.
\end{enumerate}

本システムのポイントは,プロパティに関するファセットのような元の知識ベースには存在しないファセットの生成を可能とするために,知識ベースから独立して関係データベースを設計することにある.以降で,RDFデータベース,エンティティデータベース,トリプルデータベース,ファセットデータベースを説明し,最後にプロトタイプシステムである``ProFacet''のユーザーインターフェースについて説明する.
%
\begin{figure}[h]
\centering
\includegraphics[width=3in]{image/architecture.png}
\caption{\small
システム概要
}
\label{fig:architecture}
\end{figure}
%
\subsection{RDFデータベース}
RDFデータベースは,キーワード検索とファセット検索に必要となる知識ベースのデータを格納する.本稿では,検索結果として返却されるRDFデータと検索対象となるドキュメント化されたエンティティを生成するためのRDFデータの2種類を使用する.また,検索結果として返却されるRDFデータは,エンティティのランク値の計算とファセット生成のためのクラスタリング においても使用される.
%
\subsection{エンティティデータベース}
エンティティデータベースは,エンティティを主キーとして,検索対象のドキュメント化されたエンティティ,ランク値をタプルとしたテーブルを持つ.エンティティのランク値は,検索結果として返却されるRDFデータの主語と目的語のエンティティに対して,PageRankのアルゴリズムによって計算する.この時,PageRankは,複数種類のエッジには対応しないため,元のRDFデータが持つ複数種類のプロパティの区別は行なっていない.そして,ドキュメント化されたエンティティを生成するためのRDFデータに対して,主語のエンティティと目的語のリテラルを1つのドキュメントとする処理を行う.これら2つの処理の結果を合わせて,エンティティ,ドキュメント化エンティティ,ランク値をタプルとしたエンティティテーブルを作成する.
%
\subsection{トリプルデータベース}
クラスタリング 対象のプロパティが述語であるトリプルをタプルとしたテーブルを持つ.インターフェースの検索結果には,このテーブルのタプルが表示される.
%
\subsection{ファセットデータベース}
ファセットデータベースは,エンティティに関するファセットやプロパティに関するファセットのテーブルと,各ファセットのキー名称を管理するテーブル``ファセットキー''を持つ.図~\ref{fig:architecture} では,プロパティに関するファセットを例示している.ファセットのテーブルは,ファセットの種類を示す番号,ファセットの内容を示すキー,エンティティやプロパティのURIをタプルとして持つ.ファセットキーは,ファセットの種類を示す番号,ファセットキー,キーの内容を示すラベルをタプルとして持つ.ファセットの種類を示す番号は,ファセットの種類の拡張に対応するためである.また,ファセットキーは,エンティティやプロパティのクラスタ番号に該当する.本稿では,プロパティファセットを群平均法による階層型クラスタリング によって生成した.プロパティ間の距離は,Jaccard係数を変換したJaccard距離である.Jaccard係数は2つの集合に含まれる要素のうち,共通要素が占める割合を示す.ここでは,各プロパティの主語に関するJaccard係数と目的語に関するJaccard係数の平均値をプロパティ間のJaccard係数とした.この手法を適用した理由として,プロパティはエンティティとエンティティ(またはリテラル)の関係性として機能するため,主語や目的語を共有するという観点でプロパティを整理することができると考えたからである.別の方法として,RDFスキーマ\footnote{https://www.w3.org/TR/rdf-schema}\ に基づくプロパティの階層構造を示すプロパティである``subPropertyOf''の利用や,プロパティの主語や目的語に期待されるエンティティのクラスを示すプロパティである``domain'',``range''なども考えられるが,あくまでも期待値であるため,実際のデータに基づいてクラスタリング を行うことが有効と考えた.
%
\subsection{ユーザーインターフェース}
プロパティ指向のファセット検索システム``ProFacet''のプロトタイプを実装した.そのユーザーインタフェースを図~\ref{fig:profacetInterface} に示す.Aのように,ユーザは検索キーワードやURIを入力し,主語あるいは目的語のエンティティに対して検索を実行する.検索結果はDのように主語,述語,目的語のトリプルのタプルで表示される.また,BはTransition Markerとして,Dの検索結果に対応するファセットを表示する.プロトタイプシステムでは,主語と目的語のエンティティのファセット(Subject Type, Object Type)に各エンティティのクラス情報を,述語のプロパティのファセット(Predicate Type)には上述したクラスタリング 結果を整備した.Cは,Intensionとして検索の状態を示している.

\begin{figure}[h]
\centering
\includegraphics[width=3.3in]{image/profacetInterface.png}
\caption{\small
ユーザーインターフェース
}
\label{fig:profacetInterface}
\end{figure}


\vspace{-3mm}
\section{予備実験}
\label{sec:experiments}
プロパティ指向のファセット検索システムの機能要素となるプロパティファセットの生成とその結果について目視による初期的な確認を行うことを目的として,次の2種類のプロパティのクラスタリング 結果についての確認を行なった.
\begin{enumerate}
	\item 上位プロパティによるクラスタリング
	\item 階層型クラスタリング によるクラスタリング
\end{enumerate}
また,想定するユーザ操作に従って,階層型クラスタリング によって生成したプロパティファセットによる検索結果の確認を行った.
%
\subsection{クラスタリング }
\subsubsection{上位プロパティによるクラスタリング }
知識ベースの語彙の体系に基づいて,プロパティをクラスタリング する.プロパティは,RDFスキーマと呼ばれる語彙を定義する基本的な仕組みによって,そのプロパティの性質が定義されている.その中で,上位プロパティを定義するために使用されるrdfs:subPropertyOfを利用して,各プロパティを上位プロパティでクラスタリング した.
\subsubsection{階層型クラスタリング によるクラスタリング }
実際のデータの関係性に基づいて,プロパティをクラスタリング する.トリプル集合から,プロパティ間の主語に関するJaccard係数と目的語に関するJaccard係数の平均値を計算し,群平均法による階層型クラスタリング によって,主語と目的語との関係が類似するプロパティをクラスタとしてまとめた.
%
\subsection{実験環境}
\label{sec:environment}
提案手法をPython 3.7.3で実装し,Intel(R) Core\texttrademark i7-7700 3.60 GHz CPU, 32 GB RAMを搭載したUbuntu 18.04.3 LTSで実験を行った.
%
\subsection{データセット}
\label{sec:datasets}
本実験では,DBpedia 2016-10\footnote{https://wiki.dbpedia.org/downloads-2016-10}\ のPage Linksのデータセットの約0.4\%を取得し,検索対象のエンティティとした.エンティティの数は449,339個である.そして,同じDBpedia 2016-10のMappingbased Objectsのデータセットから検索対象のエンティティが含まれるトリプルを抽出し,クラスタリング対象のデータセットとした.トリプルに含まれるプロパティの種類数は617個である.
%
\subsection{上位プロパティによるクラスタリング 結果}
\label{sec:upperPropertyClusteringResult}
上位プロパティを使うと,617個のプロパティのうち,615個のプロパティを41個のクラスタにまとめることができた.2個のプロパティは,上位プロパティが定義されていないため,クラスタから漏れてしまった.図~\ref{fig:upperPropertyClusters}は41個のクラスタから,プロパティ数が10個以上となったクラスタの一覧である.約4割のプロパティが,2つの上位プロパティにまとめられていることがわかる.それらのクラスタには,多様な領域に関するプロパティがメンバーとして含まれていることを確認した.

また,図~\ref{fig:partOfUpperPropertyCluster}は,``hasPart'' クラスタに含まれるプロパティの一覧である.組織,都市,食物など,様々な領域に関するプロパティが含まれていることがわかる.
%
\begin{figure}[h]
\centering
\includegraphics[width=3in]{image/upperPropertyClusters.png}
\caption{\small
上位プロパティによるクラスタリング 結果の一部(プロパティ数が10以上のクラスタの一覧)
}
\label{fig:upperPropertyClusters}
\end{figure}
%
\begin{figure}[h]
\centering
\includegraphics[scale=0.5]{image/partOfUpperPropertyCluster.png}
\caption{\small
``hasPart''クラスタのプロパティ一覧
}
\label{fig:partOfUpperPropertyCluster}
\end{figure}
%
\subsection{階層型クラスタリング によるクラスタリング 結果}
\label{sec:hierachicalClustersResult}
階層型クラスタリング では,617個のプロパティを128個のクラスタを生成した.クラスタの生成はデンドログラムを確認し,分割の閾値を0.995として実行した.図~\ref{fig:hierachicalClusters}は,プロパティ数が10個以上となったクラスタの一覧で19個ある.クラスタの名称は,クラスタ内のプロパティの内容を確認して便宜的に付けた.クラスタのサイズは最大でも51となり,上位プロパティによるクラスタリング と比較して,クラスタのサイズの分散は小さくなっていることは明らかである.クラスタ内のプロパティが関係する領域も,クラスタの名称通り,一定の傾向があることを確認した.図~\ref{fig:partOfHierachicalCluster}を具体例として提示する.一定の領域に関係するプロパティの集合となっていることが確認できる.
%
\begin{figure}[h]
\centering
\includegraphics[width=3in]{image/hierachicalClusters.png}
\caption{\small
階層型クラスタリング によるクラスタリング 結果の一部(プロパティ数が10以上のクラスタの一覧)
}
\label{fig:hierachicalClusters}
\end{figure}
%
\begin{figure}[h]
\centering
\includegraphics[scale=0.5]{image/partOfHierachicalCluster.png}
\caption{\small
``競走馬に関するプロパティ''クラスタのプロパティ一覧
}
\label{fig:partOfHierachicalCluster}
\end{figure}
%
\subsection{プロパティファセットによる検索結果}
実験に関して想定するユーザは次の通りである.
\begin{enumerate}
	\item 知識ベースに関する専門知識を持たない
	\item ``Arab Spring''について調べる
\end{enumerate}

\subsubsection{キーワード検索結果}
キーワード検索の結果,主語のエンティティのドキュメントに``Arab Spring''を含むトリプルを373個,目的語のエンティティのドキュメントに``Arab Spring''を含むトリプルを245個を得た.全トリプルに含まれるプロパティの種類数は68個だった.図~\ref{fig:keywordResult}はその一部の抜粋である.
%
\begin{figure}[h]
\centering
\includegraphics[width=3.5in]{image/keywordResult.png}
\caption{\small
``Arab Spring''のキーワード検索結果
}
\label{fig:keywordResult}
\end{figure}
%
\subsubsection{キーワード検索結果に対応するプロパティファセット}
キーワード検索結果に対応するプロパティファセットは,図~\ref{fig:propertyFacet}の通りである.なお,この結果は図~\ref{fig:hierachicalClusters}で提示したファセットキーのみに絞っている.この結果を踏まえて,ユーザは政治に関するプロパティを選択すると仮定する.
%
\begin{figure}[h]
\centering
\includegraphics[width=3in]{image/propertyFacet.png}
\caption{\small
キーワード検索結果に対応するプロパティファセット
}
\label{fig:propertyFacet}
\end{figure}
%
\subsubsection{ファセット検索結果}
政治に関するプロパティのキーを選択した検索結果は,図~\ref{fig:facetResult}の通りである.主語のエンティティのドキュメントに``Arab Spring''を含むトリプルを10個,目的語のエンティティのドキュメントに``Arab Spring''を含むトリプルを11個得た.``Arab Spring''に関係する国や政党の指導者や思想を記述したトリプルであることがわかる.
%
\begin{figure}[h]
\centering
\includegraphics[width=3.5in]{image/facetResult.png}
\caption{\small
``Arab Spring''のキーワード検索結果に対するファセット検索結果
}
\label{fig:facetResult}
\end{figure}
%




\vspace{-3mm}
\section{まとめと今後の課題}
本稿では,プロパティ指向のファセット検索システムの全体概要について提案し,その機能要素となるファセット生成のためのクラスタリング に関する予備実験について報告した.予備実験では,プロパティの主語と目的語に関するJaccard係数を利用した階層型クラスタリング によって,一定程度関係する領域でプロパティをまとめられることを確認できた.また,上位プロパティを利用したクラスタリング の場合は,同一クラスタであっても関係する領域の異なるプロパティの集合になることを確認できた.

今後の課題として,これらのクラスタリング を使った検索のユーザビリティについて調査を行う.また,そのためにプロパティ指向のファセット検索システムの実装についても進める必要がある.

\vspace{-3mm}
\section{謝辞}
2019年度共同研究 (SKY株式会社) (CPE01017) 「データエンジニアリングの知見の応用によるSKYSEA Client Viewのログ及び資産情報の処理の高速化・軽量化・高度化」の助成を受けたものです。

\vspace{5mm}
\bibliographystyle{jplain}
%%\bibliographystyle{ACM-Reference-Format}
{\footnotesize % フォントサイズ
\bibliography{reference}
}

\end{document}
